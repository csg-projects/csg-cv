\documentclass[a4paper,10pt]{article} % Default font size and paper size

%{{{ Packages & Includes
%\usepackage{xunicode,xltxtra,url,parskip} % Formatting packages
\usepackage{url,parskip} % Formatting packages
\usepackage{graphicx}
\usepackage{adjustbox}

\usepackage[usenames,dvipsnames]{xcolor} % Required for specifying custom colors

\usepackage[left=2cm,right=2cm,top=2cm,bottom=2cm]{geometry}
\usepackage{hyperref} % Required for adding links	and customizing them
\definecolor{linkcolour}{rgb}{0,0.2,0.6} % Link color
\hypersetup{colorlinks,breaklinks,urlcolor=linkcolour,linkcolor=linkcolour} % Set link colors throughout the document
\usepackage{array}
\newcolumntype{R}[1]{>{\raggedleft\let\newline\\\arraybackslash\hspace{0pt}}m{#1}}
\newcolumntype{L}[1]{>{\raggedright\let\newline\\\arraybackslash\hspace{0pt}}m{#1}}
\newcolumntype{C}[1]{>{\centering\let\newline\\\arraybackslash\hspace{0pt}}m{#1}}

\usepackage{titlesec} % Used to customize the \section command
\titleformat{\section}{\scshape\raggedright}{}{0em}{}[\titlerule] % Text formatting of sections

\titlespacing{\section}{0pt}{8pt}{3pt} % Spacing around sections
%}}}

% Variable Definition
\newcommand\rightColumnWidth{13.55cm}
\newcommand\leftColumnWidth{1.75cm}
\newcommand\pageWidth{16cm}

\begin{document}

\pagestyle{empty} % Removes page numbering

\section{Personal Information}

\begin{table}[ht]
    \begin{tabular}{L{1.75cm}p{3cm}p{1cm}L{1.4cm}L{1.5cm}L{0.5cm}L{1.5cm}L{3cm}}
        Full Name: & \textbf{Carlos Segarra} & & \multicolumn{2}{l}{\textbf{Languages:}}  & & \multicolumn{2}{l}{\textbf{Relevant links:}}\\
        Birth year: & 1996 & & Spanish: & Native & & Github: & \href{https://github.com/csegarragonz}{csegarragonz} \\
        Residence: & London, UK & & English: & Proficient & & LinkedIn: & \href{https://linkedin.com/in/carlossegrrag}{carlossegarrag} \\
        Mail: & \small{\href{mailto:carlos@carlossegarra.com}{carlos@carlossegarra.com}} & & French: & Fluent & & Web: & \href{https://carlossegarra.com}{carlossegarra.com} \\
        Phone: & +44 7309347084
    \end{tabular}
\end{table}

\section{Short Bio}
\begin{tabular}{p{\pageWidth}}
    I am a third year PhD student at the Large-Scale Data and Systems Group (LSDS) of the Imperial College London, under the supervision of Prof. Peter Pietzuch.
    My research addresses the design and implementation of secure, high-performance, runtime systems for the cloud.
    I have a special interest in privacy-preserving systems and confidential computing.
    As part of my PhD, I have interned with Intel Labs in the datacenter security team.
    Before joining Imperial, I received an MSc in advanced mathematics, a BSc in mathematics, and a BSc in electrical engineering from the Technical University of Catalonia (UPC).
    In the past, I have worked as a researcher for the Barcelona Supercomputing Center (BSC), Nokia Bell Labs, the Swiss Center for Electronics and Microtechnology (CSEM), the University of Neuchatel and the UPC.
\end{tabular}

\section{Publications}
\begin{tabular}{R{\leftColumnWidth}p{\rightColumnWidth}}
    \textit{Submitted}   & TLess: A Confidential Serverless Runtime with Attestation and Authorisation \\
                            & \textbf{C. Segarra}, S. Shillaker, D. Goltzsche, A. Vahldiek-Oberwagner, M. Steiner, M. Vij, L. Vilanova, R. Kapitza, and P. Pietzuch \\[3pt]
    \textit{Submitted}   & Granny: Fine-Grained Distribution of Scientific Workloads in the Cloud \\
                            & S. Shillaker, \textbf{C. Segarra}, M. Mathys, M. Fournial, and P. Pietzuch \\[3pt]
    Bitcoin'21  & A Novel Framework for the Analysis of Unknown Transactions in Bitcoin: Theory, Model, and Experimental Results \\
                & M. Caprolu, M. Pontecorvi, M. Signorini,
                \textbf{C. Segarra}, and R. Di Pietro \\[3pt]
    EuroSys'20  & KOLLAPS: Decentralized and Dynamic Topology Emulator \\
            & P. Gouveia, J. Neves, \textbf{C. Segarra}, L. Liechti, S. Issa, V. Schiavoni, and M. Matos \\[3pt]
    SRDS'20 & Hardening IoT Brokers Using ARM TrustZone for Secure Pub/Sub Middleware \\
            & \textbf{C. Segarra}, and R. Delgado-Gonzalo, and V. Schiavoni \\[3pt]
    MIE'20  & MQT-TZ: Secure MQTT Broker for Biomedical Signal Processing on the Edge \\
            & \textbf{C. Segarra}, and R. Delgado-Gonzalo, and V. Schiavoni \\[3pt]
    DAIS'19 & Using Trusted Execution Environments for Secure Stream Processing of Medical Data \\
            & \textbf{C. Segarra}, M. Lemay, R. Delgado-Gonzalo, and V. Schiavoni, \\[3pt]
    EMBC'19 & Secure Stream Processing for Medical Data \\
            & \textbf{C. Segarra}, E. Muntan\'e, M. Lemay, V. Schiavoni, and  R. Delgado-Gonzalo \\[3pt]
\end{tabular}

\section{Education}

\begin{tabular}{R{\leftColumnWidth}|p{\rightColumnWidth}}
    \emph{Current} & PhD in Computing \\
    \textsc{10/2020} & Imperial College London - Sup. Professor Peter Pietzuch \\
\end{tabular}

\begin{tabular}{R{\leftColumnWidth}|p{\rightColumnWidth}}
    \textsc{07/2020} & Master in \textbf{\textsc{Advanced Mathematics and Mathematical Engineering}} \\
    \textsc{09/2019} & \small{\emph{Technical University of Catalonia}, UPC}\\
     % & \footnotesize{MSc in Advanced Mathematics with a focus in Discrete Mathematics and Algorithms. Enrolled to courses from the Master in Research in Informatics (MIRI-UPC). Relevant courses cover: Graph Theory, Codes and Cryptography, and Concurrence, Parallelism, and Distributed Systems.}
\end{tabular}

\begin{tabular}{R{\leftColumnWidth}|p{\rightColumnWidth}}
    \textsc{05/2019} & Bachelor's degree in \textbf{\textsc{Mathematics}} \\
    \textsc{09/2014} & \small{\emph{Technical University of Catalonia}, UPC}\\
     % & \footnotesize{BSc in Mathematics within a double degree program in the Interdisciplinary Higher Education Center (CFIS) at the UPC. Relevant coursework covers real and complex analysis, statistics, probability and graph theory, combinatorics and algorithms.}
\end{tabular}

\begin{tabular}{R{\leftColumnWidth}|p{\rightColumnWidth}}
    \textsc{05/2019} &  Bachelor`s degree in \textbf{\textsc{Telecommunications Science and Technology}}\\
    \textsc{09/2014} & \small{\emph{Technical University of Catalonia}, UPC} \\
     % & \footnotesize{BSc in Telecommunications Engineering within a double degree program in the Interdisciplinary Higher Education Center (CFIS) at the UPC. Relevant coursework covers networking and concurrency, digital and analogical communications and real time digital signal processing.}
\end{tabular}

\section{Work Experience}
\begin{tabular}{R{\leftColumnWidth}|p{\rightColumnWidth}}
    \textsc{06/2022} & Graduate Research Intern at \textbf{\textsc{Intel Labs}} \\
    \textsc{09/2022} & \small{\emph{Datacenter Security Group} - Sup. Mona Vij}\\
    % & \footnotesize{Research collaborator with the Computer Architecture Department (DAC) of the School of Informatics (FIB) of the Technical University of Catalonia (UPC) in Barcelona, Spain. Under the supervision of Jordi Guitart, we work on live migration of Docker containers with the goal of performing live migrations of container clusters. To perform migrations we rely on Checkpoint Restore In Userspace (CRIU) technology.}
\end{tabular}

\begin{tabular}{R{\leftColumnWidth}|p{\rightColumnWidth}}
    \textsc{06/2020} & Research Collaborator at the \textbf{\textsc{Technical University of Catalonia}} \\
    \textsc{10/2019} & \small{\emph{Computer Architecture Department} - Sup. Jordi Guitart, PhD }\\
    % & \footnotesize{Research collaborator with the Computer Architecture Department (DAC) of the School of Informatics (FIB) of the Technical University of Catalonia (UPC) in Barcelona, Spain. Under the supervision of Jordi Guitart, we work on live migration of Docker containers with the goal of performing live migrations of container clusters. To perform migrations we rely on Checkpoint Restore In Userspace (CRIU) technology.}
\end{tabular}

\begin{tabular}{R{\leftColumnWidth}|p{\rightColumnWidth}}
    \textsc{02/2020} & Research Assistant at the \textbf{\textsc{Universit\'e de Neuch\^atel}} \\
    \textsc{08/2019} & \small{\emph{Complex Systems Group} - Sup. Valerio Schiavoni, PhD }\\
    % & \footnotesize{Research collaborator with the Complex Systems Group from the UniNe. Currently working on an emulation platform for distributed applications. We simulate arbitrary topologies basing on the end-to-end properties of the links among the different nodes in the network, and dynamic link behaviours using traffic control functionalities available in the Linux Kernel. For the emulated deployment we rely on Docker containers and orchestrators such as Swarm and Kubernetes.}
\end{tabular}

\begin{tabular}{R{\leftColumnWidth}|p{\rightColumnWidth}}
    \textsc{07/2019} & Trainee at the \textbf{\textsc{Swiss Center for Electronics and Microtechnology} (CSEM)} \\
    \textsc{10/2018} & \small{\emph{Embedded Software Division} - Sup. Ricard Delgado Gonzalo, PhD }\\
    % & \footnotesize{Intern at the Embedded Software Division of the CSEM in Neuch\^atel, CH, using Trusted Execution Environments to perform privacy-preserving computations on IoT medical devices. Developed a distributed streaming platform on Intel SGX and a secure implementation of the MQTT broker relying on TLS and ARM TrustZone. Collaborator in two H2020 EU Projects: ACTIVAGE and TABEDE. In the former, responsible of implementing part of the Security \& Privacy module and in the latter responsible of implementing the User Control Interface, visualizing data coming from a variety of IoT devices.}
\end{tabular}

\begin{tabular}{R{\leftColumnWidth}|p{\rightColumnWidth}}
    \textsc{09/2018} & Trainee at \textbf{\textsc{Nokia Bell Labs}} \\
    \textsc{06/2018} & \small{\emph{Security Group} - Sup. Matteo Signorini, PhD and Matteo Pontecorvi, PhD}\\
% & \footnotesize{Summer intern at the the Cybersecurity department at Nokia Bell Labs in Paris-Saclay, France. I developed a graph-based model to detect, cluster and classify chains of malicious transactions in the Bitcoin's Blockchain. By defining an abstraction layer atop the chain and an isomorphism class, we managed to identify a variety of malicious services.}
\end{tabular}

\begin{tabular}{R{\leftColumnWidth}|p{\rightColumnWidth}}
    \textsc{06/2018} & Research Student at the \textbf{\textsc{Barcelona Supercomputing Center} (BSC)} \\
    \textsc{04/2017} & \small{\emph{Workflows and Distributed Computing Group} - Sup. Rosa M. Badia, PhD} \\
    % & \footnotesize{Research student at the Workflows and Distributed Computing Group at the BSC in Barcelona, Spain. I developed, deployed and benchmarked a distributed implementation of the DBSCAN clustering algorithm using COMP Superscalar, a programming model for distributed computing. Evaluation was done in the \textit{Mare Nostrum} supercomputer.}
\end{tabular}

\end{document}
